\documentclass{coderdojopotsdam}

\title{Teilnahmebedingungen}

\begin{document}
\begin{Form}

	Dieses Formular muss \textbf{vor Teilnahme} des Kindes/Jugendlichen von einem Erziehungsberechtigten ausgefüllt und \textbf{unterschrieben} werden.

	\textfield{7.5cm}{Name des Kindes/der Kinder}

	\textfield{7.5cm}{Name des Erziehungsberechtigten}

	\textfield{7.5cm}{Telefonnummer für den Notfall}

	\textfield{7.5cm}{E-Mail-Adresse}

	\section*{Mentoren und Technologie}

	Ich habe verstanden und akzeptiere, dass:

	\begin{itemize}
		\item[$\blacksquare$] alle Coder-Dojo-Mentoren ihre Aufgabe freiwillig und nach ihren Fähigkeiten erfüllen.
			Die Veranstalter des Coder Dojos Potsdam haben keine vertragliche Beziehung zu den Mentoren und tragen keine Haftung für deren Verhalten, sondern sorgen lediglich für den organisatorischen Rahmen.
			Die Zusammenarbeit erfolgt auf Vertrauensbasis.
		\item[$\blacksquare$] weder die Mentoren noch der Veranstalter für irgendwelche Schäden oder Verluste haften, die sich direkt oder indirekt aus den Ratschlägen und Instruktionen, oder dem Ausbleiben von Ratschlägen und Instruktionen, der Mentoren ergeben, es sei denn, der Schaden wurde grob fahrlässig oder vorsätzlich hervorgerufen.
		\item[$\blacksquare$] mein Kind – und gegebenenfalls ich selbst – an diesem Workshop freiwillig und auf eigenes Risiko teilnehmen.
			Wir entscheiden selbst, ob wir den Hilfestellungen der Mentoren folgen oder nicht.
			Eine Haftung der Veranstalter oder der Mentoren ist ausgeschlossen.
			Wenn beispielsweise der mitgebrachte Computer auf Grund von aus dem Internet geladener oder zur Verfügung gestellter Software abstürzt oder anderen Schaden nimmt, dann tragen wir als Teilnehmer die volle Verantwortung.
			Weder beim Veranstalter noch bei den Mentoren kann Schadenersatz geltend gemacht werden.
	\end{itemize}

	\section*{Verantwortlichkeiten der Teilnehmer}

	Jedes bis zu zehnjährige Kind muss von einem Erziehungsberechtigten begleitet werden, der sich für das Kind einschreibt und für das Kind während der Durchführung des Coder Dojos Potsdam verantwortlich ist.
	Der Erziehungsberechtigte muss während der gesamten Zeit der Durchführung der Veranstaltung anwesend sein, Ausnahmen sind nicht gestattet.

	Es liegt in der Verantwortung der Eltern/Erziehungsberechtigten, die angemessene Beaufsichtigung aller Kinder und Jugendlichen in der Gruppe der 11- bis 18-Jährigen sicherzustellen.
	Das Coder Dojo Potsdam übernimmt keinerlei Haftung oder andere Verantwortung für die Kinder; derlei Verantwortung verbleibt zu allen Zeitpunkten der Durchführung des Coder Dojos Potsdam bei den Eltern/Erziehungsberechtigten.
	Die teilnehmenden Kinder haben den Anweisungen der Mentoren Folge zu leisten.

	Sollte ein unbegleitetes teilnehmendes Kind während der Durchführung des Coder Dojos erkranken, so werden die Mentoren alle Möglichkeiten nutzen, die Eltern/Erziehungsberechtigten zu verständigen, damit dem Kind angemessene medizinische Hilfe zukommen kann.
	In extremen Fällen, in denen medizinische Hilfe ohne zeitlichen Verzug notwendig und es unmöglich ist, den oben genannten Kontakt zu verständigen, gebe ich hiermit \textbf{den Mentoren} das Einverständnis, dem Kind die notwendige medizinische Hilfe zukommen zu lassen.
	Die Mentoren und der Veranstalter werden von der Haftung für Schäden durch Notfallmaßnahmen ausgeschlossen.

	\textbf{Wir stimmen hiermit der Teilnahme zu obigen Bedingungen zu.}

	\signature

	\emph{Die unterschriebene Seite mitbringen oder an \raisebox{-2pt}{\includegraphics{email.pdf}} senden.}

\end{Form}
\end{document}
